%%% LaTeX Template: Two column article
%%%
%%% Source: http://www.howtotex.com/
%%% Feel free to distribute this template, but please keep to referal to http://www.howtotex.com/ here.
%%% Date: February 2011

%%% Preamble
\documentclass[	DIV=calc,%
							paper=a4,%
							fontsize=12pt,%
							onecolumn]{scrartcl}	 					% KOMA-article class

\usepackage{lipsum}													% Package to create dummy text
\usepackage[brazil]{babel}										% English language/hyphenation
\usepackage[protrusion=true,expansion=true]{microtype}				% Better typography
\usepackage{amsmath,amsfonts,amsthm}					% Math packages
\usepackage[pdftex]{graphicx}									% Enable pdflatex
\usepackage[svgnames]{xcolor}									% Enabling colors by their 'svgnames'
\usepackage[hang, small,labelfont=bf,up,textfont=it,up]{caption}	% Custom captions under/above floats
\usepackage{epstopdf}												% Converts .eps to .pdf
\usepackage{subfig}													% Subfigures
\usepackage{booktabs}												% Nicer tables
\usepackage{fix-cm}													% Custom fontsizes
\usepackage[utf8]{inputenc}
\usepackage[top=2.5cm, bottom=2.5cm, left=2.5cm, right=2.5cm]{geometry}
\usepackage[ddmmyyyy]{datetime}
\addto\captionsenglish{%
	\renewcommand\tablename{Tabela}
	\renewcommand\figurename{Figura}
} 
 

 
%%% Custom sectioning (sectsty package)
\usepackage{sectsty}													% Custom sectioning (see below)
\allsectionsfont{%															% Change font of al section commands
	\usefont{OT1}{phv}{b}{n}%										% bch-b-n: CharterBT-Bold font
	}

\sectionfont{%																% Change font of \section command
	\usefont{OT1}{phv}{b}{n}%										% bch-b-n: CharterBT-Bold font
	}



%%% Headers and footers
\usepackage{fancyhdr}												% Needed to define custom headers/footers
	\pagestyle{fancy}														% Enabling the custom headers/footers
\usepackage{lastpage}	

% Header (empty)
\lhead{}
\chead{}
\rhead{}
% Footer (you may change this to your own needs)

%% ====================================
%% ====================================
%% mude o rodape  do projeto
%% ====================================
%% ====================================

\lfoot{\footnotesize \texttt{Especificação do processo} }


\cfoot{}
\rfoot{\footnotesize página \thepage\ de \pageref{LastPage}}	% "Page 1 of 2"
\renewcommand{\headrulewidth}{0.0pt}
\renewcommand{\footrulewidth}{0.4pt}



%%% Creating an initial of the very first character of the content
\usepackage{lettrine}
\newcommand{\initial}[1]{%
     \lettrine[lines=3,lhang=0.3,nindent=0em]{
     				\color{DarkGoldenrod}
     				{\textsf{#1}}}{}}



%%% Title, author and date metadata
\usepackage{titling}															% For custom titles

\newcommand{\HorRule}{\color{DarkGoldenrod}%			% Creating a horizontal rule
									  	\rule{\linewidth}{1pt}%
										}

\pretitle{\vspace{-30pt} \begin{flushleft} \HorRule 
				\fontsize{50}{50} \usefont{OT1}{phv}{b}{n} \color{DarkRed} \selectfont 
				}

%% ====================================
%% ====================================
%% mude o titulo  do projeto
%% ====================================
%% ====================================

\title{Especificação do processo de gerenciamento de versões}					% Title of your article goes here

%% ====================================



\posttitle{\par\end{flushleft}\vskip 0.5em}

\preauthor{\begin{flushleft}
					\large \lineskip 0.5em \usefont{OT1}{phv}{b}{sl} \color{DarkRed}}
\author{Diego, Emerson, Jonata e Vinícius }  	% Author name goes here


\postauthor{\footnotesize \usefont{OT1}{phv}{m}{sl} \color{Black} 
					\\Universidade Tecnológica Federal do Paraná - Câmpus Cornélio Procópio 								% Institution of author
					\par\end{flushleft}\HorRule}

\date{}																				% No date




%%% Begin document
\begin{document}
\maketitle
\thispagestyle{fancy} 	
\thispagestyle{empty}		% Enabling the custom headers/footers for the first page 
% The first character should be within \initial{}




%% ====================================
%% ====================================
%% mude o resumo  do projeto
%% ====================================
%% ====================================
\initial{A}\textbf{adoção de soluções modernas e com qualidade depende de um processo estabelecido que possa assegurar que o software a
ser desenvolvido atenda às necessidades do cliente. Para que isso ocorra é necessário a
definição de todos os processos envolvidos e como eles serão interligados.
Este documento exemplifica o processo de desenvolvimento de software a ser aplicado 
na empresa Software Supimpa Tecnologia (2ST), que deseja manter um processo para o gerenciamento de versões do software Klassic, que já se encontra no mercado. Para isso, o processo foi definido seguindo a Norma ISO/IEC 12207 e a prática DevOps. }

%% ====================================
\begin{figure}
	\centering
	\includegraphics{utfpr}
\end{figure}

\vspace{3cm}
\centerline{\textit{\textbf{\today}}}

\clearpage
    \renewcommand*\listfigurename{Lista de figuras}
\listoffigures

\renewcommand*\listtablename{Lista de tabelas}
\listoftables




\clearpage
\renewcommand{\contentsname}{Sumário}
\tableofcontents
\clearpage

%% ====================================
%% ====================================
%% Inicio do texto
%% ====================================
%% ====================================
\section{Escopo}

Este documento tem por objetivo definir a utilização da Baseline da aplicação Klassic da 
empresa Software Supimpa Tecnologia (2ST)
Através deste é possível o desenvolvedor se iterar sobre a aplicação e todas as
tecnologias, ferramentas, códigos e demais necessário para utilizar e continuar o desenvolvimento da
aplicação e suas novas versões.
Neste documento também é abordado o processo de instalação do ambiente de
desenvolvimento, assim caso necessite substituir e/ou adicionar outro equipamento sem ter problemas
com este quesito. 


\section{Tecnologias e ferramentas utilizadas}
\begin{enumerate}
	\item Linguagem Java
	\item NetBeans 8
	\item PostgreSQL
	\item Java Persistence API (JPA)
	\item Astah
\end{enumerate}



\section{Instalação}
O projeto utiliza algumas tecnologias especificas (citadas
anteriormente) para que possa funcionar. Além disso, faz necessários o download da Baseline direto do
repositório e algumas configurações para que o mesmo inicialize.
Abaixo será descrito o passo a passo para que o projeto seja iniciado em um novo computador,
este com sistema operacional Linux, Windows ou MacOS. 

Acesse o link do repositório do projeto, este hospedado no GitHub atualmente, onde nele você tem acesso a toda a documentação do sistema.

https://github.com/viniciusaugutis/ProdutoGerencia

Após o acesso, faça o download o projeto para sua máquina.
Em seguida instale o Java (linguagem de programação utilizada). Segue o link para download: $https://www.java.com/pt_BR/download/$

Também é preciso que se instale o PostgreSQL que é o banco de dados relacional utilizado na aplicação. Segue o link para download: https://www.postgresql.org/download/

Também é preciso instalar a IDE utilizada para desenvolver o sistema, que no caso é o NetBeans 8. Segue o link para download:
https://netbeans.org/downloads/


\section{Versionamento}

O versionamento da aplicação é essencial para um projeto que tende a crescer a curto ou longo
prazo. Ele registra alterações em um arquivo ou conjunto de arquivos ao longo do tempo para que você
possa lembrar versões específicas mais tarde.
Para o controle de versionamento da aplicação é utilizada a ferramenta GIT. Cada diretório de
trabalho do Git é um repositório com um histórico completo e habilidade total de acompanhamento das
revisões, não dependente de acesso a uma rede ou a um servidor central.
Como plataforma de hospedagem de código para controle de versão e colaboração é utilizado o
GitHub, que é um dos mais populares na atualidade. Ele inovou ao criar uma aplicação web dinâmica,
com uma interface agradável e que conseguiu tornar a contribuição de software através um processo
fácil e sociável.
O git funciona em vários sistemas operacionais, como Mac, Windows e Linux. A primeira tarefa
do desenvolvedor consiste em instalar o git na sua máquina de trabalho. A seguir segue o processo de
instalação: https://git-scm.com/book/en/v2/Getting-Started-Installing-Git
Abaixo serão explicados alguns comandos básicos e iniciais do GIT para começar a trabalhar com
o controle de versionamento.


\begin{enumerate}
	\item git init: iniciar um repositório git
	\item git add nome-do-arquivo-incluindo-extensão: usado para adicionar um arquivo para controle
	\item git status: mostra o estado do repositório, sendo quais arquivos estão fora do controle, quais
	foram modificados e estão esperando por uma descrição de modificação
	\item git commit -m "Mensagem do commit": comita arquivos do repositório, onde comitar é salvar
	uma alteração de código-fonte em um sistema de versionamento.
	\item git reset HEAD nome-do-arquivo: volta ao estágio anterior ao commit
\end{enumerate}

Abaixo serão explicados alguns comandos do GIT que o desenvolver deve saber para começar a
trabalhar com a nossa aplicação.


\begin{enumerate}
	\item git clone url-do-projeto: faz um clone do nosso repositório remoto da aplicação para sua
	máquina.
	\item git pull: faz uma sincronização com os arquivos mais atualizados do repositório remoto.
	\item git push origin nomeBranch: envia modificações da sua máquina local para o repositório
	remoto
	\item git checkout -b nome-do-branch: cria um branch no projeto. Os branchs são ramificações da
	árvore central do seu projeto, onde eu posso criar vários branchs e após a realização da tarefa eu
	posso juntar ele novamente ao branch principal.
\end{enumerate}

No link a seguir está a documentação completa do git e que se faz necessário o desenvolvedor
aprender: https://git-scm.com/docs/git
Para desenvolvimento da nossa aplicação o processo consiste através de pulls-requests. Um pullrequest
permite que você informe as mudanças que você enviou para um repositório no GitHub. Uma
vez que um pedido é aberto, você pode discutir e rever as possíveis mudanças com os colaboradores,
para depois disso o revisor aceitar ou não o seu pedido.
Para isso você pode usar o comando git request-pull 1.0.0 url-projeto master.
O número do release varia de acordo com o que foi feito pelo desenvolvedor. Se foi consertado
algum bug o último número deve ser incrementado (1.0.1). Se foi adicionada uma nova funcionalidade o
segudo número deve ser incrementado (1.1.0). Se foi feita uma mudança que alterou a compatibilidade
com a versão anterior da baseline, o primeiro número deve ser incrementado (2.0.0)

\section{Padrões}

Existem dois padrões de escrita:

Snake Case: Define que as palavras são separadas com um caractere de sublinhado  e sem espaços,
com a letra inicial de cada elemento geralmente em minúsculas dentro do composto. Exemplo: foo_bar.

Camel Case: Define que as palavras compostas ou frase devem ser iniciadas com maiúsculas e unidas
sem espaços. Exemplo: FooBar


\section{Classes e métodos}
Os usuários do processo são todos os envolvidos com o desenvolvimento de software. Dentre eles: desenvolvedor, gerente do projeto, operador de infraestrutura, analistas de sistemas e designers.




\end{document}